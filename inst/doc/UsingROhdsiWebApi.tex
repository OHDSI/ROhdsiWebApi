% Options for packages loaded elsewhere
\PassOptionsToPackage{unicode}{hyperref}
\PassOptionsToPackage{hyphens}{url}
%
\documentclass[
]{article}
\usepackage{lmodern}
\usepackage{amssymb,amsmath}
\usepackage{ifxetex,ifluatex}
\ifnum 0\ifxetex 1\fi\ifluatex 1\fi=0 % if pdftex
  \usepackage[T1]{fontenc}
  \usepackage[utf8]{inputenc}
  \usepackage{textcomp} % provide euro and other symbols
\else % if luatex or xetex
  \usepackage{unicode-math}
  \defaultfontfeatures{Scale=MatchLowercase}
  \defaultfontfeatures[\rmfamily]{Ligatures=TeX,Scale=1}
\fi
% Use upquote if available, for straight quotes in verbatim environments
\IfFileExists{upquote.sty}{\usepackage{upquote}}{}
\IfFileExists{microtype.sty}{% use microtype if available
  \usepackage[]{microtype}
  \UseMicrotypeSet[protrusion]{basicmath} % disable protrusion for tt fonts
}{}
\makeatletter
\@ifundefined{KOMAClassName}{% if non-KOMA class
  \IfFileExists{parskip.sty}{%
    \usepackage{parskip}
  }{% else
    \setlength{\parindent}{0pt}
    \setlength{\parskip}{6pt plus 2pt minus 1pt}}
}{% if KOMA class
  \KOMAoptions{parskip=half}}
\makeatother
\usepackage{xcolor}
\IfFileExists{xurl.sty}{\usepackage{xurl}}{} % add URL line breaks if available
\IfFileExists{bookmark.sty}{\usepackage{bookmark}}{\usepackage{hyperref}}
\hypersetup{
  pdftitle={Using ROhdsiWebApi},
  pdfauthor={Gowtham Rao},
  hidelinks,
  pdfcreator={LaTeX via pandoc}}
\urlstyle{same} % disable monospaced font for URLs
\usepackage[margin=1in]{geometry}
\usepackage{color}
\usepackage{fancyvrb}
\newcommand{\VerbBar}{|}
\newcommand{\VERB}{\Verb[commandchars=\\\{\}]}
\DefineVerbatimEnvironment{Highlighting}{Verbatim}{commandchars=\\\{\}}
% Add ',fontsize=\small' for more characters per line
\usepackage{framed}
\definecolor{shadecolor}{RGB}{248,248,248}
\newenvironment{Shaded}{\begin{snugshade}}{\end{snugshade}}
\newcommand{\AlertTok}[1]{\textcolor[rgb]{0.94,0.16,0.16}{#1}}
\newcommand{\AnnotationTok}[1]{\textcolor[rgb]{0.56,0.35,0.01}{\textbf{\textit{#1}}}}
\newcommand{\AttributeTok}[1]{\textcolor[rgb]{0.77,0.63,0.00}{#1}}
\newcommand{\BaseNTok}[1]{\textcolor[rgb]{0.00,0.00,0.81}{#1}}
\newcommand{\BuiltInTok}[1]{#1}
\newcommand{\CharTok}[1]{\textcolor[rgb]{0.31,0.60,0.02}{#1}}
\newcommand{\CommentTok}[1]{\textcolor[rgb]{0.56,0.35,0.01}{\textit{#1}}}
\newcommand{\CommentVarTok}[1]{\textcolor[rgb]{0.56,0.35,0.01}{\textbf{\textit{#1}}}}
\newcommand{\ConstantTok}[1]{\textcolor[rgb]{0.00,0.00,0.00}{#1}}
\newcommand{\ControlFlowTok}[1]{\textcolor[rgb]{0.13,0.29,0.53}{\textbf{#1}}}
\newcommand{\DataTypeTok}[1]{\textcolor[rgb]{0.13,0.29,0.53}{#1}}
\newcommand{\DecValTok}[1]{\textcolor[rgb]{0.00,0.00,0.81}{#1}}
\newcommand{\DocumentationTok}[1]{\textcolor[rgb]{0.56,0.35,0.01}{\textbf{\textit{#1}}}}
\newcommand{\ErrorTok}[1]{\textcolor[rgb]{0.64,0.00,0.00}{\textbf{#1}}}
\newcommand{\ExtensionTok}[1]{#1}
\newcommand{\FloatTok}[1]{\textcolor[rgb]{0.00,0.00,0.81}{#1}}
\newcommand{\FunctionTok}[1]{\textcolor[rgb]{0.00,0.00,0.00}{#1}}
\newcommand{\ImportTok}[1]{#1}
\newcommand{\InformationTok}[1]{\textcolor[rgb]{0.56,0.35,0.01}{\textbf{\textit{#1}}}}
\newcommand{\KeywordTok}[1]{\textcolor[rgb]{0.13,0.29,0.53}{\textbf{#1}}}
\newcommand{\NormalTok}[1]{#1}
\newcommand{\OperatorTok}[1]{\textcolor[rgb]{0.81,0.36,0.00}{\textbf{#1}}}
\newcommand{\OtherTok}[1]{\textcolor[rgb]{0.56,0.35,0.01}{#1}}
\newcommand{\PreprocessorTok}[1]{\textcolor[rgb]{0.56,0.35,0.01}{\textit{#1}}}
\newcommand{\RegionMarkerTok}[1]{#1}
\newcommand{\SpecialCharTok}[1]{\textcolor[rgb]{0.00,0.00,0.00}{#1}}
\newcommand{\SpecialStringTok}[1]{\textcolor[rgb]{0.31,0.60,0.02}{#1}}
\newcommand{\StringTok}[1]{\textcolor[rgb]{0.31,0.60,0.02}{#1}}
\newcommand{\VariableTok}[1]{\textcolor[rgb]{0.00,0.00,0.00}{#1}}
\newcommand{\VerbatimStringTok}[1]{\textcolor[rgb]{0.31,0.60,0.02}{#1}}
\newcommand{\WarningTok}[1]{\textcolor[rgb]{0.56,0.35,0.01}{\textbf{\textit{#1}}}}
\usepackage{longtable,booktabs}
% Correct order of tables after \paragraph or \subparagraph
\usepackage{etoolbox}
\makeatletter
\patchcmd\longtable{\par}{\if@noskipsec\mbox{}\fi\par}{}{}
\makeatother
% Allow footnotes in longtable head/foot
\IfFileExists{footnotehyper.sty}{\usepackage{footnotehyper}}{\usepackage{footnote}}
\makesavenoteenv{longtable}
\usepackage{graphicx,grffile}
\makeatletter
\def\maxwidth{\ifdim\Gin@nat@width>\linewidth\linewidth\else\Gin@nat@width\fi}
\def\maxheight{\ifdim\Gin@nat@height>\textheight\textheight\else\Gin@nat@height\fi}
\makeatother
% Scale images if necessary, so that they will not overflow the page
% margins by default, and it is still possible to overwrite the defaults
% using explicit options in \includegraphics[width, height, ...]{}
\setkeys{Gin}{width=\maxwidth,height=\maxheight,keepaspectratio}
% Set default figure placement to htbp
\makeatletter
\def\fps@figure{htbp}
\makeatother
\setlength{\emergencystretch}{3em} % prevent overfull lines
\providecommand{\tightlist}{%
  \setlength{\itemsep}{0pt}\setlength{\parskip}{0pt}}
\setcounter{secnumdepth}{5}

\title{Using ROhdsiWebApi}
\author{Gowtham Rao}
\date{2020-09-17}

\begin{document}
\maketitle

{
\setcounter{tocdepth}{2}
\tableofcontents
}
ROhdsiWebApi is part of \href{https://ohdsi.github.io/Hades}{HADES}.

\hypertarget{introduction}{%
\section{Introduction}\label{introduction}}

From Package
\href{https://github.com/OHDSI/ROhdsiWebApi/blob/master/README.md}{Readme}

\begin{quote}
ROhdsiWebApi is a R based interface to
\href{https://github.com/ohdsi/webapi}{`WebApi'} (OHDSI RESTful
services), and performs GET/PULL/POST/DELETE calls via the WebApi. All
objects starting from R or output to R - are analysis ready R-objects
like list and data.frame. The package handles the intermediary steps by
converting R-objects to JSON and vice versa. To ensure r-objects are
analysis ready, the objects are type converted where possible,
e.g.~date/date time are converted from string to POSIXct.
\end{quote}

\begin{quote}
This package makes reproducible research easier, by offering ability to
retrieve detailed study specifications, transport study specifications
from one instance to another, programmatically invoke the generation of
a sequence of steps that are part of a study, manage running studies in
batch mode.
\end{quote}

This document will attempt to explain how ROhdsiWebApi maybe used to
achieve reproducible research.

\hypertarget{webapi-configurations-and-rohdsiwebapi}{%
\section{WebApi configurations and
ROhdsiWebApi}\label{webapi-configurations-and-rohdsiwebapi}}

To successfully use ROhdsiWebApi, it is necessary to have an active
\href{https://github.com/ohdsi/webapi}{`WebApi'} endpoint with a known
baseUrl such as ``\url{http://server.org:80/WebAPI}''.
\href{https://github.com/ohdsi/webapi}{`WebApi'} has many functional
categories.

To ensure reproducibility of work it is best to know the version of the
WebApi (i.e.~Atlas backend) being used. An easy way to do that is (and
output maybe included in your study results)

\begin{Shaded}
\begin{Highlighting}[]

\NormalTok{version <-}\StringTok{ }\NormalTok{ROhdsiWebApi}\OperatorTok{:::}\KeywordTok{getWebApiVersion}\NormalTok{(}\DataTypeTok{baseUrl =}\NormalTok{ baseUrl)}
\NormalTok{message1 <-}\StringTok{ }\KeywordTok{paste0}\NormalTok{(}\StringTok{'This Vignette was created using WebApi version: '}\NormalTok{, }
\NormalTok{                   version, }
                   \StringTok{' on baseUrl: '}\NormalTok{, }
\NormalTok{                   baseUrl, }
                   \StringTok{". The CDM had the following source data configured: "}\NormalTok{)}
\NormalTok{cdmSources <-}\StringTok{ }\NormalTok{ROhdsiWebApi}\OperatorTok{::}\KeywordTok{getCdmSources}\NormalTok{(}\DataTypeTok{baseUrl =}\NormalTok{ baseUrl)}
\NormalTok{priorityVocabulary <-}\StringTok{ }\NormalTok{ROhdsiWebApi}\OperatorTok{::}\KeywordTok{getPriorityVocabularyKey}\NormalTok{(}\DataTypeTok{baseUrl =}\NormalTok{ baseUrl)}
\end{Highlighting}
\end{Shaded}

The object \texttt{version} will show your webApi version. Example: This
Vignette was created using WebApi version: 2.7.7 on baseUrl:
\url{http://api.ohdsi.org:80/WebAPI}. The CDM had the following source
data configured: .

\begin{Shaded}
\begin{Highlighting}[]

\NormalTok{cdmSources}
\CommentTok{#> # A tibble: 3 x 7}
\CommentTok{#>   sourceId sourceName            sourceKey  sourceDialect cdmDatabaseSchema vocabDatabaseSchema resultsDatabaseSchema}
\CommentTok{#>      <int> <chr>                 <chr>      <chr>         <chr>             <chr>               <chr>                }
\CommentTok{#> 1        4 Common Evidence Model CEM        postgresql    <NA>              unrestricted        <NA>                 }
\CommentTok{#> 2        6 SYNPUF 1K             SYNPUF1K   postgresql    synpuf1k          unrestricted        synpuf1k_results     }
\CommentTok{#> 3        5 SYNPUF 5%             SYNPUF5PCT postgresql    synpuf5pct        unrestricted        synpuf5pct_results}
\end{Highlighting}
\end{Shaded}

The priority vocabulary for the WebApi is SYNPUF5PCT.

We can also perform checks on the WebApi, example - we may want to see
if the `HCUP' \& `SYNPUF1K' is a valid SourceKey in the current webApi.

\begin{Shaded}
\begin{Highlighting}[]

\NormalTok{ROhdsiWebApi}\OperatorTok{::}\KeywordTok{isValidSourceKey}\NormalTok{(}\DataTypeTok{sourceKeys =} \KeywordTok{c}\NormalTok{(}\StringTok{'HCUP'}\NormalTok{, }\StringTok{'SYNPUF1K'}\NormalTok{), }\DataTypeTok{baseUrl =}\NormalTok{ baseUrl)}
\CommentTok{#> [1] FALSE  TRUE}
\end{Highlighting}
\end{Shaded}

\hypertarget{webapi-analytical-categories.}{%
\subsection{WebApi Analytical
categories.}\label{webapi-analytical-categories.}}

WebApi maybe considered to have certain modular analytic categories.
ROhdsiWebApi supports the following categories:

\begin{longtable}[]{@{}ll@{}}
\toprule
\begin{minipage}[b]{0.21\columnwidth}\raggedright
Category\strut
\end{minipage} & \begin{minipage}[b]{0.73\columnwidth}\raggedright
Features\strut
\end{minipage}\tabularnewline
\midrule
\endhead
\begin{minipage}[t]{0.21\columnwidth}\raggedright
ConceptSet\strut
\end{minipage} & \begin{minipage}[t]{0.73\columnwidth}\raggedright
Functions for interfacing with ConceptSet in WebApi\strut
\end{minipage}\tabularnewline
\begin{minipage}[t]{0.21\columnwidth}\raggedright
Cohort\strut
\end{minipage} & \begin{minipage}[t]{0.73\columnwidth}\raggedright
Functions for interfacing with Cohort in WebApi\strut
\end{minipage}\tabularnewline
\begin{minipage}[t]{0.21\columnwidth}\raggedright
IncidenceRate\strut
\end{minipage} & \begin{minipage}[t]{0.73\columnwidth}\raggedright
Functions for interfacing with IncidenceRate in WebApi\strut
\end{minipage}\tabularnewline
\begin{minipage}[t]{0.21\columnwidth}\raggedright
Estimation\strut
\end{minipage} & \begin{minipage}[t]{0.73\columnwidth}\raggedright
Functions for interfacing with Estimation in WebApi\strut
\end{minipage}\tabularnewline
\begin{minipage}[t]{0.21\columnwidth}\raggedright
Prediction\strut
\end{minipage} & \begin{minipage}[t]{0.73\columnwidth}\raggedright
Functions for interfacing with Prediction in WebApi\strut
\end{minipage}\tabularnewline
\begin{minipage}[t]{0.21\columnwidth}\raggedright
Characterization\strut
\end{minipage} & \begin{minipage}[t]{0.73\columnwidth}\raggedright
Functions for interfacing with Characterization in WebApi\strut
\end{minipage}\tabularnewline
\begin{minipage}[t]{0.21\columnwidth}\raggedright
Pathway\strut
\end{minipage} & \begin{minipage}[t]{0.73\columnwidth}\raggedright
Functions for interfacing with Pathway in WebApi\strut
\end{minipage}\tabularnewline
\bottomrule
\end{longtable}

\hypertarget{framework-of-rohdsiwebapi}{%
\section{Framework of ROhdsiWebApi}\label{framework-of-rohdsiwebapi}}

ROhdsiWebApi maybe better understood by having atleast a high level
understanding of
\href{https://en.wikipedia.org/wiki/Create,_read,_update_and_delete}{CRUD}
framework for \href{https://github.com/ohdsi/webapi}{WebApi}, i.e.~the
GET, PUT, DELETE, POST calls to the API. See
\href{http://webapidoc.ohdsi.org/index.html}{the documentation of the
WebApi}.

For each supported category, ROhdsiWebAPi performs GET, PUT, DELETE,
POST calls to WebApi in background. The details of what calls are
actually performed is less important to an analyst, but it is useful to
understand the naming conventions of ROhdsiWebApi.

\hypertarget{naming-conventions-of-rohdsiwebapi}{%
\subsection{Naming conventions of
ROhdsiWebApi}\label{naming-conventions-of-rohdsiwebapi}}

Most functions in ROhdsiWebApi start with an action oriented `verb' -
such as

\begin{longtable}[]{@{}ll@{}}
\toprule
\begin{minipage}[b]{0.46\columnwidth}\raggedright
Function Name\strut
\end{minipage} & \begin{minipage}[b]{0.48\columnwidth}\raggedright
Description\strut
\end{minipage}\tabularnewline
\midrule
\endhead
\begin{minipage}[t]{0.46\columnwidth}\raggedright
cancelCharacterizationGeneration\strut
\end{minipage} & \begin{minipage}[t]{0.48\columnwidth}\raggedright
Cancel Characterization Generation\strut
\end{minipage}\tabularnewline
\begin{minipage}[t]{0.46\columnwidth}\raggedright
cancelCohortGeneration\strut
\end{minipage} & \begin{minipage}[t]{0.48\columnwidth}\raggedright
Cancel Cohort Generation\strut
\end{minipage}\tabularnewline
\begin{minipage}[t]{0.46\columnwidth}\raggedright
cancelGeneration\strut
\end{minipage} & \begin{minipage}[t]{0.48\columnwidth}\raggedright
Cancel Generation\strut
\end{minipage}\tabularnewline
\begin{minipage}[t]{0.46\columnwidth}\raggedright
cancelIncidenceRateGeneration\strut
\end{minipage} & \begin{minipage}[t]{0.48\columnwidth}\raggedright
Cancel Incidence Rate Generation\strut
\end{minipage}\tabularnewline
\begin{minipage}[t]{0.46\columnwidth}\raggedright
cancelPathwayGeneration\strut
\end{minipage} & \begin{minipage}[t]{0.48\columnwidth}\raggedright
Cancel Pathway Generation\strut
\end{minipage}\tabularnewline
\begin{minipage}[t]{0.46\columnwidth}\raggedright
convertConceptSetDefinitionToTable\strut
\end{minipage} & \begin{minipage}[t]{0.48\columnwidth}\raggedright
Convert Concept Set Definition To Table\strut
\end{minipage}\tabularnewline
\begin{minipage}[t]{0.46\columnwidth}\raggedright
createConceptSetWorkbook\strut
\end{minipage} & \begin{minipage}[t]{0.48\columnwidth}\raggedright
Create Concept Set Workbook\strut
\end{minipage}\tabularnewline
\begin{minipage}[t]{0.46\columnwidth}\raggedright
deleteCharacterizationDefinition\strut
\end{minipage} & \begin{minipage}[t]{0.48\columnwidth}\raggedright
Delete Characterization Definition\strut
\end{minipage}\tabularnewline
\begin{minipage}[t]{0.46\columnwidth}\raggedright
deleteCohortDefinition\strut
\end{minipage} & \begin{minipage}[t]{0.48\columnwidth}\raggedright
Delete Cohort Definition\strut
\end{minipage}\tabularnewline
\begin{minipage}[t]{0.46\columnwidth}\raggedright
deleteConceptSetDefinition\strut
\end{minipage} & \begin{minipage}[t]{0.48\columnwidth}\raggedright
Delete Concept Set Definition\strut
\end{minipage}\tabularnewline
\begin{minipage}[t]{0.46\columnwidth}\raggedright
deleteDefinition\strut
\end{minipage} & \begin{minipage}[t]{0.48\columnwidth}\raggedright
Delete Definition\strut
\end{minipage}\tabularnewline
\begin{minipage}[t]{0.46\columnwidth}\raggedright
deleteEstimationDefinition\strut
\end{minipage} & \begin{minipage}[t]{0.48\columnwidth}\raggedright
Delete Estimation Definition\strut
\end{minipage}\tabularnewline
\begin{minipage}[t]{0.46\columnwidth}\raggedright
deleteIncidenceRateDefinition\strut
\end{minipage} & \begin{minipage}[t]{0.48\columnwidth}\raggedright
Delete Incidence Rate Definition\strut
\end{minipage}\tabularnewline
\begin{minipage}[t]{0.46\columnwidth}\raggedright
deletePathwayDefinition\strut
\end{minipage} & \begin{minipage}[t]{0.48\columnwidth}\raggedright
Delete Pathway Definition\strut
\end{minipage}\tabularnewline
\begin{minipage}[t]{0.46\columnwidth}\raggedright
deletePredictionDefinition\strut
\end{minipage} & \begin{minipage}[t]{0.48\columnwidth}\raggedright
Delete Prediction Definition\strut
\end{minipage}\tabularnewline
\begin{minipage}[t]{0.46\columnwidth}\raggedright
detectCharacterizationsByName\strut
\end{minipage} & \begin{minipage}[t]{0.48\columnwidth}\raggedright
Detect Characterizations By Name\strut
\end{minipage}\tabularnewline
\begin{minipage}[t]{0.46\columnwidth}\raggedright
detectCohortsByName\strut
\end{minipage} & \begin{minipage}[t]{0.48\columnwidth}\raggedright
Detect Cohorts By Name\strut
\end{minipage}\tabularnewline
\begin{minipage}[t]{0.46\columnwidth}\raggedright
detectConceptSetsByName\strut
\end{minipage} & \begin{minipage}[t]{0.48\columnwidth}\raggedright
Detect Concept Sets By Name\strut
\end{minipage}\tabularnewline
\begin{minipage}[t]{0.46\columnwidth}\raggedright
detectEstimationsByName\strut
\end{minipage} & \begin{minipage}[t]{0.48\columnwidth}\raggedright
Detect Estimations By Name\strut
\end{minipage}\tabularnewline
\begin{minipage}[t]{0.46\columnwidth}\raggedright
detectIncidenceRatesByName\strut
\end{minipage} & \begin{minipage}[t]{0.48\columnwidth}\raggedright
Detect Incidence Rates By Name\strut
\end{minipage}\tabularnewline
\begin{minipage}[t]{0.46\columnwidth}\raggedright
detectPathwaysByName\strut
\end{minipage} & \begin{minipage}[t]{0.48\columnwidth}\raggedright
Detect Pathways By Name\strut
\end{minipage}\tabularnewline
\begin{minipage}[t]{0.46\columnwidth}\raggedright
detectPredictionsByName\strut
\end{minipage} & \begin{minipage}[t]{0.48\columnwidth}\raggedright
Detect Predictions By Name\strut
\end{minipage}\tabularnewline
\begin{minipage}[t]{0.46\columnwidth}\raggedright
existsCharacterizationName\strut
\end{minipage} & \begin{minipage}[t]{0.48\columnwidth}\raggedright
Exists Characterization Name\strut
\end{minipage}\tabularnewline
\begin{minipage}[t]{0.46\columnwidth}\raggedright
existsCohortName\strut
\end{minipage} & \begin{minipage}[t]{0.48\columnwidth}\raggedright
Exists Cohort Name\strut
\end{minipage}\tabularnewline
\begin{minipage}[t]{0.46\columnwidth}\raggedright
existsConceptSetName\strut
\end{minipage} & \begin{minipage}[t]{0.48\columnwidth}\raggedright
Exists Concept Set Name\strut
\end{minipage}\tabularnewline
\begin{minipage}[t]{0.46\columnwidth}\raggedright
existsEstimationName\strut
\end{minipage} & \begin{minipage}[t]{0.48\columnwidth}\raggedright
Exists Estimation Name\strut
\end{minipage}\tabularnewline
\begin{minipage}[t]{0.46\columnwidth}\raggedright
existsIncidenceRateName\strut
\end{minipage} & \begin{minipage}[t]{0.48\columnwidth}\raggedright
Exists Incidence Rate Name\strut
\end{minipage}\tabularnewline
\begin{minipage}[t]{0.46\columnwidth}\raggedright
existsPathwayName\strut
\end{minipage} & \begin{minipage}[t]{0.48\columnwidth}\raggedright
Exists Pathway Name\strut
\end{minipage}\tabularnewline
\begin{minipage}[t]{0.46\columnwidth}\raggedright
existsPredictionName\strut
\end{minipage} & \begin{minipage}[t]{0.48\columnwidth}\raggedright
Exists Prediction Name\strut
\end{minipage}\tabularnewline
\begin{minipage}[t]{0.46\columnwidth}\raggedright
getCdmsources\strut
\end{minipage} & \begin{minipage}[t]{0.48\columnwidth}\raggedright
Get Cdmsources\strut
\end{minipage}\tabularnewline
\begin{minipage}[t]{0.46\columnwidth}\raggedright
getCharacterizationDefinition\strut
\end{minipage} & \begin{minipage}[t]{0.48\columnwidth}\raggedright
Get Characterization Definition\strut
\end{minipage}\tabularnewline
\begin{minipage}[t]{0.46\columnwidth}\raggedright
getCharacterizationDefinitionsMetadata\strut
\end{minipage} & \begin{minipage}[t]{0.48\columnwidth}\raggedright
Get Characterization Definitions Metadata\strut
\end{minipage}\tabularnewline
\begin{minipage}[t]{0.46\columnwidth}\raggedright
getCharacterizationGenerationinformation\strut
\end{minipage} & \begin{minipage}[t]{0.48\columnwidth}\raggedright
Get Characterization Generationinformation\strut
\end{minipage}\tabularnewline
\begin{minipage}[t]{0.46\columnwidth}\raggedright
getCharacterizationResults\strut
\end{minipage} & \begin{minipage}[t]{0.48\columnwidth}\raggedright
Get Characterization Results\strut
\end{minipage}\tabularnewline
\begin{minipage}[t]{0.46\columnwidth}\raggedright
getCohortDefinition\strut
\end{minipage} & \begin{minipage}[t]{0.48\columnwidth}\raggedright
Get Cohort Definition\strut
\end{minipage}\tabularnewline
\begin{minipage}[t]{0.46\columnwidth}\raggedright
getCohortDefinitionExpression\strut
\end{minipage} & \begin{minipage}[t]{0.48\columnwidth}\raggedright
Get Cohort Definition Expression\strut
\end{minipage}\tabularnewline
\begin{minipage}[t]{0.46\columnwidth}\raggedright
getCohortDefinitionName\strut
\end{minipage} & \begin{minipage}[t]{0.48\columnwidth}\raggedright
Get Cohort Definition Name\strut
\end{minipage}\tabularnewline
\begin{minipage}[t]{0.46\columnwidth}\raggedright
getCohortDefinitionSql\strut
\end{minipage} & \begin{minipage}[t]{0.48\columnwidth}\raggedright
Get Cohort Definition Sql\strut
\end{minipage}\tabularnewline
\begin{minipage}[t]{0.46\columnwidth}\raggedright
getCohortDefinitionsMetadata\strut
\end{minipage} & \begin{minipage}[t]{0.48\columnwidth}\raggedright
Get Cohort Definitions Metadata\strut
\end{minipage}\tabularnewline
\begin{minipage}[t]{0.46\columnwidth}\raggedright
getCohortGenerationinformation\strut
\end{minipage} & \begin{minipage}[t]{0.48\columnwidth}\raggedright
Get Cohort Generationinformation\strut
\end{minipage}\tabularnewline
\begin{minipage}[t]{0.46\columnwidth}\raggedright
getCohortInclusionrulesandcounts\strut
\end{minipage} & \begin{minipage}[t]{0.48\columnwidth}\raggedright
Get Cohort Inclusionrulesandcounts\strut
\end{minipage}\tabularnewline
\begin{minipage}[t]{0.46\columnwidth}\raggedright
getCohortResults\strut
\end{minipage} & \begin{minipage}[t]{0.48\columnwidth}\raggedright
Get Cohort Results\strut
\end{minipage}\tabularnewline
\begin{minipage}[t]{0.46\columnwidth}\raggedright
getCohortSql\strut
\end{minipage} & \begin{minipage}[t]{0.48\columnwidth}\raggedright
Get Cohort Sql\strut
\end{minipage}\tabularnewline
\begin{minipage}[t]{0.46\columnwidth}\raggedright
getConceptSetDefinition\strut
\end{minipage} & \begin{minipage}[t]{0.48\columnwidth}\raggedright
Get Concept Set Definition\strut
\end{minipage}\tabularnewline
\begin{minipage}[t]{0.46\columnwidth}\raggedright
getConceptSetDefinitionsMetadata\strut
\end{minipage} & \begin{minipage}[t]{0.48\columnwidth}\raggedright
Get Concept Set Definitions Metadata\strut
\end{minipage}\tabularnewline
\begin{minipage}[t]{0.46\columnwidth}\raggedright
getConceptSetsFromCohortDefinition\strut
\end{minipage} & \begin{minipage}[t]{0.48\columnwidth}\raggedright
Get Concept Sets From Cohort Definition\strut
\end{minipage}\tabularnewline
\begin{minipage}[t]{0.46\columnwidth}\raggedright
getConcepts\strut
\end{minipage} & \begin{minipage}[t]{0.48\columnwidth}\raggedright
Get Concepts\strut
\end{minipage}\tabularnewline
\begin{minipage}[t]{0.46\columnwidth}\raggedright
getDefinition\strut
\end{minipage} & \begin{minipage}[t]{0.48\columnwidth}\raggedright
Get Definition\strut
\end{minipage}\tabularnewline
\begin{minipage}[t]{0.46\columnwidth}\raggedright
getDefinitionsMetadata\strut
\end{minipage} & \begin{minipage}[t]{0.48\columnwidth}\raggedright
Get Definitions Metadata\strut
\end{minipage}\tabularnewline
\begin{minipage}[t]{0.46\columnwidth}\raggedright
getEstimationDefinition\strut
\end{minipage} & \begin{minipage}[t]{0.48\columnwidth}\raggedright
Get Estimation Definition\strut
\end{minipage}\tabularnewline
\begin{minipage}[t]{0.46\columnwidth}\raggedright
getEstimationDefinitionsMetadata\strut
\end{minipage} & \begin{minipage}[t]{0.48\columnwidth}\raggedright
Get Estimation Definitions Metadata\strut
\end{minipage}\tabularnewline
\begin{minipage}[t]{0.46\columnwidth}\raggedright
getGenerationinformation\strut
\end{minipage} & \begin{minipage}[t]{0.48\columnwidth}\raggedright
Get Generationinformation\strut
\end{minipage}\tabularnewline
\begin{minipage}[t]{0.46\columnwidth}\raggedright
getIncidenceRateDefinition\strut
\end{minipage} & \begin{minipage}[t]{0.48\columnwidth}\raggedright
Get Incidence Rate Definition\strut
\end{minipage}\tabularnewline
\begin{minipage}[t]{0.46\columnwidth}\raggedright
getIncidenceRateDefinitionsMetadata\strut
\end{minipage} & \begin{minipage}[t]{0.48\columnwidth}\raggedright
Get Incidence Rate Definitions Metadata\strut
\end{minipage}\tabularnewline
\begin{minipage}[t]{0.46\columnwidth}\raggedright
getIncidenceRateGenerationinformation\strut
\end{minipage} & \begin{minipage}[t]{0.48\columnwidth}\raggedright
Get Incidence Rate Generationinformation\strut
\end{minipage}\tabularnewline
\begin{minipage}[t]{0.46\columnwidth}\raggedright
getIncidenceRateResults\strut
\end{minipage} & \begin{minipage}[t]{0.48\columnwidth}\raggedright
Get Incidence Rate Results\strut
\end{minipage}\tabularnewline
\begin{minipage}[t]{0.46\columnwidth}\raggedright
getPathwayDefinition\strut
\end{minipage} & \begin{minipage}[t]{0.48\columnwidth}\raggedright
Get Pathway Definition\strut
\end{minipage}\tabularnewline
\begin{minipage}[t]{0.46\columnwidth}\raggedright
getPathwayDefinitionsMetadata\strut
\end{minipage} & \begin{minipage}[t]{0.48\columnwidth}\raggedright
Get Pathway Definitions Metadata\strut
\end{minipage}\tabularnewline
\begin{minipage}[t]{0.46\columnwidth}\raggedright
getPathwayGenerationinformation\strut
\end{minipage} & \begin{minipage}[t]{0.48\columnwidth}\raggedright
Get Pathway Generationinformation\strut
\end{minipage}\tabularnewline
\begin{minipage}[t]{0.46\columnwidth}\raggedright
getPathwayResults\strut
\end{minipage} & \begin{minipage}[t]{0.48\columnwidth}\raggedright
Get Pathway Results\strut
\end{minipage}\tabularnewline
\begin{minipage}[t]{0.46\columnwidth}\raggedright
getPersonProfile\strut
\end{minipage} & \begin{minipage}[t]{0.48\columnwidth}\raggedright
Get Person Profile\strut
\end{minipage}\tabularnewline
\begin{minipage}[t]{0.46\columnwidth}\raggedright
getPredictionDefinition\strut
\end{minipage} & \begin{minipage}[t]{0.48\columnwidth}\raggedright
Get Prediction Definition\strut
\end{minipage}\tabularnewline
\begin{minipage}[t]{0.46\columnwidth}\raggedright
getPredictionDefinitionsMetadata\strut
\end{minipage} & \begin{minipage}[t]{0.48\columnwidth}\raggedright
Get Prediction Definitions Metadata\strut
\end{minipage}\tabularnewline
\begin{minipage}[t]{0.46\columnwidth}\raggedright
getPriorityvocabularykey\strut
\end{minipage} & \begin{minipage}[t]{0.48\columnwidth}\raggedright
Get Priorityvocabularykey\strut
\end{minipage}\tabularnewline
\begin{minipage}[t]{0.46\columnwidth}\raggedright
getResults\strut
\end{minipage} & \begin{minipage}[t]{0.48\columnwidth}\raggedright
Get Results\strut
\end{minipage}\tabularnewline
\begin{minipage}[t]{0.46\columnwidth}\raggedright
getSourceconcepts\strut
\end{minipage} & \begin{minipage}[t]{0.48\columnwidth}\raggedright
Get Sourceconcepts\strut
\end{minipage}\tabularnewline
\begin{minipage}[t]{0.46\columnwidth}\raggedright
getWebApiVersion\strut
\end{minipage} & \begin{minipage}[t]{0.48\columnwidth}\raggedright
Get Web Api Version\strut
\end{minipage}\tabularnewline
\begin{minipage}[t]{0.46\columnwidth}\raggedright
insertCohortDefinitionInPackage\strut
\end{minipage} & \begin{minipage}[t]{0.48\columnwidth}\raggedright
Insert Cohort Definition In Package\strut
\end{minipage}\tabularnewline
\begin{minipage}[t]{0.46\columnwidth}\raggedright
insertCohortDefinitionSetInPackage\strut
\end{minipage} & \begin{minipage}[t]{0.48\columnwidth}\raggedright
Insert Cohort Definition Set In Package\strut
\end{minipage}\tabularnewline
\begin{minipage}[t]{0.46\columnwidth}\raggedright
invokeCharacterizationGeneration\strut
\end{minipage} & \begin{minipage}[t]{0.48\columnwidth}\raggedright
Invoke Characterization Generation\strut
\end{minipage}\tabularnewline
\begin{minipage}[t]{0.46\columnwidth}\raggedright
invokeCohortGeneration\strut
\end{minipage} & \begin{minipage}[t]{0.48\columnwidth}\raggedright
Invoke Cohort Generation\strut
\end{minipage}\tabularnewline
\begin{minipage}[t]{0.46\columnwidth}\raggedright
invokeGeneration\strut
\end{minipage} & \begin{minipage}[t]{0.48\columnwidth}\raggedright
Invoke Generation\strut
\end{minipage}\tabularnewline
\begin{minipage}[t]{0.46\columnwidth}\raggedright
invokeIncidenceRateGeneration\strut
\end{minipage} & \begin{minipage}[t]{0.48\columnwidth}\raggedright
Invoke Incidence Rate Generation\strut
\end{minipage}\tabularnewline
\begin{minipage}[t]{0.46\columnwidth}\raggedright
invokePathwayGeneration\strut
\end{minipage} & \begin{minipage}[t]{0.48\columnwidth}\raggedright
Invoke Pathway Generation\strut
\end{minipage}\tabularnewline
\begin{minipage}[t]{0.46\columnwidth}\raggedright
isvalidCharacterizationId\strut
\end{minipage} & \begin{minipage}[t]{0.48\columnwidth}\raggedright
Isvalid Characterization Id\strut
\end{minipage}\tabularnewline
\begin{minipage}[t]{0.46\columnwidth}\raggedright
isvalidCohortId\strut
\end{minipage} & \begin{minipage}[t]{0.48\columnwidth}\raggedright
Isvalid Cohort Id\strut
\end{minipage}\tabularnewline
\begin{minipage}[t]{0.46\columnwidth}\raggedright
isvalidConceptSetId\strut
\end{minipage} & \begin{minipage}[t]{0.48\columnwidth}\raggedright
Isvalid Concept Set Id\strut
\end{minipage}\tabularnewline
\begin{minipage}[t]{0.46\columnwidth}\raggedright
isvalidEstimationId\strut
\end{minipage} & \begin{minipage}[t]{0.48\columnwidth}\raggedright
Isvalid Estimation Id\strut
\end{minipage}\tabularnewline
\begin{minipage}[t]{0.46\columnwidth}\raggedright
isvalidId\strut
\end{minipage} & \begin{minipage}[t]{0.48\columnwidth}\raggedright
Isvalid Id\strut
\end{minipage}\tabularnewline
\begin{minipage}[t]{0.46\columnwidth}\raggedright
isvalidIncidenceRateId\strut
\end{minipage} & \begin{minipage}[t]{0.48\columnwidth}\raggedright
Isvalid Incidence Rate Id\strut
\end{minipage}\tabularnewline
\begin{minipage}[t]{0.46\columnwidth}\raggedright
isvalidPathwayId\strut
\end{minipage} & \begin{minipage}[t]{0.48\columnwidth}\raggedright
Isvalid Pathway Id\strut
\end{minipage}\tabularnewline
\begin{minipage}[t]{0.46\columnwidth}\raggedright
isvalidPredictionId\strut
\end{minipage} & \begin{minipage}[t]{0.48\columnwidth}\raggedright
Isvalid Prediction Id\strut
\end{minipage}\tabularnewline
\begin{minipage}[t]{0.46\columnwidth}\raggedright
isvalidSourceKey\strut
\end{minipage} & \begin{minipage}[t]{0.48\columnwidth}\raggedright
Isvalid Source Key\strut
\end{minipage}\tabularnewline
\begin{minipage}[t]{0.46\columnwidth}\raggedright
postCharacterizationDefinition\strut
\end{minipage} & \begin{minipage}[t]{0.48\columnwidth}\raggedright
Post Characterization Definition\strut
\end{minipage}\tabularnewline
\begin{minipage}[t]{0.46\columnwidth}\raggedright
postCohortDefinition\strut
\end{minipage} & \begin{minipage}[t]{0.48\columnwidth}\raggedright
Post Cohort Definition\strut
\end{minipage}\tabularnewline
\begin{minipage}[t]{0.46\columnwidth}\raggedright
postConceptSetDefinition\strut
\end{minipage} & \begin{minipage}[t]{0.48\columnwidth}\raggedright
Post Concept Set Definition\strut
\end{minipage}\tabularnewline
\begin{minipage}[t]{0.46\columnwidth}\raggedright
postDefinition\strut
\end{minipage} & \begin{minipage}[t]{0.48\columnwidth}\raggedright
Post Definition\strut
\end{minipage}\tabularnewline
\begin{minipage}[t]{0.46\columnwidth}\raggedright
postEstimationDefinition\strut
\end{minipage} & \begin{minipage}[t]{0.48\columnwidth}\raggedright
Post Estimation Definition\strut
\end{minipage}\tabularnewline
\begin{minipage}[t]{0.46\columnwidth}\raggedright
postIncidenceRateDefinition\strut
\end{minipage} & \begin{minipage}[t]{0.48\columnwidth}\raggedright
Post Incidence Rate Definition\strut
\end{minipage}\tabularnewline
\begin{minipage}[t]{0.46\columnwidth}\raggedright
postPathwayDefinition\strut
\end{minipage} & \begin{minipage}[t]{0.48\columnwidth}\raggedright
Post Pathway Definition\strut
\end{minipage}\tabularnewline
\begin{minipage}[t]{0.46\columnwidth}\raggedright
postPredictionDefinition\strut
\end{minipage} & \begin{minipage}[t]{0.48\columnwidth}\raggedright
Post Prediction Definition\strut
\end{minipage}\tabularnewline
\begin{minipage}[t]{0.46\columnwidth}\raggedright
resolveConceptSet\strut
\end{minipage} & \begin{minipage}[t]{0.48\columnwidth}\raggedright
Resolve Concept Set\strut
\end{minipage}\tabularnewline
\bottomrule
\end{longtable}

Most of the functions start with the following verbs:

\begin{longtable}[]{@{}lr@{}}
\toprule
Function Verb & Number Of Functions\tabularnewline
\midrule
\endhead
Get & 38\tabularnewline
Isvalid & 9\tabularnewline
Delete & 8\tabularnewline
Post & 8\tabularnewline
Detect & 7\tabularnewline
Exists & 7\tabularnewline
Cancel & 5\tabularnewline
Invoke & 5\tabularnewline
Insert & 2\tabularnewline
Convert & 1\tabularnewline
Create & 1\tabularnewline
Resolve & 1\tabularnewline
\bottomrule
\end{longtable}

A function to get \texttt{Definition} is \texttt{getDefinitionMetaData}
function. This is a general function that is able to get the Metadata
for all specifications within a category.

\begin{Shaded}
\begin{Highlighting}[]

\NormalTok{ROhdsiWebApi}\OperatorTok{::}\KeywordTok{getDefinitionsMetadata}\NormalTok{(}\DataTypeTok{baseUrl =}\NormalTok{ baseUrl, }
                                     \DataTypeTok{category =} \StringTok{'cohort'}\NormalTok{) }\OperatorTok
\StringTok{  }\KeywordTok{arrange}\NormalTok{(.data}\OperatorTok{$}\NormalTok{id) }\OperatorTok
\StringTok{  }\KeywordTok{rename_all}\NormalTok{(}\DataTypeTok{.funs =}\NormalTok{ SqlRender}\OperatorTok{::}\NormalTok{camelCaseToTitleCase) }\OperatorTok
\StringTok{  }\KeywordTok{tail}\NormalTok{() }
\CommentTok{#> # A tibble: 6 x 7}
\CommentTok{#>        Id Name                         `Created By` `Created Date`      `Modified By` Description `Modified Date`    }
\CommentTok{#>     <int> <chr>                        <chr>        <dttm>              <chr>         <chr>       <dttm>             }
\CommentTok{#> 1 1775307 "CDM Research Exercise 4"    ""           2015-03-06 20:49:00 ""             <NA>       NA                 }
\CommentTok{#> 2 1775308 "Metformin user"             ""           2015-03-06 20:49:00 ""             <NA>       NA                 }
\CommentTok{#> 3 1775309 "Sulfonylurea user"          ""           2015-03-06 20:49:00 ""             <NA>       NA                 }
\CommentTok{#> 4 1775310 "COPY OF: Sulfonylurea user" ""           2015-03-06 20:49:00 ""             <NA>       NA                 }
\CommentTok{#> 5 1775311 "Hypoglycemia "              ""           2015-03-06 20:49:00 ""             <NA>       NA                 }
\CommentTok{#> 6 1775316 "AF (FA_17092020)"           ""           2015-03-06 20:49:00 ""            ""          NA}
\end{Highlighting}
\end{Shaded}

The same output may be achieved using

\begin{Shaded}
\begin{Highlighting}[]

\NormalTok{ROhdsiWebApi}\OperatorTok{::}\KeywordTok{getCohortDefinitionsMetaData}\NormalTok{(}\DataTypeTok{baseUrl =}\NormalTok{ baseUrl) }\OperatorTok\StringTok{ }
\StringTok{  }\KeywordTok{arrange}\NormalTok{(.data}\OperatorTok{$}\NormalTok{id) }\OperatorTok\StringTok{ }
\StringTok{  }\KeywordTok{rename_all}\NormalTok{(}\DataTypeTok{.funs =}\NormalTok{ SqlRender}\OperatorTok{::}\NormalTok{camelCaseToTitleCase) }\OperatorTok\StringTok{ }
\StringTok{  }\KeywordTok{tail}\NormalTok{() }
\CommentTok{#> # A tibble: 6 x 7}
\CommentTok{#>        Id Name                         `Created By` `Created Date`      `Modified By` Description `Modified Date`    }
\CommentTok{#>     <int> <chr>                        <chr>        <dttm>              <chr>         <chr>       <dttm>             }
\CommentTok{#> 1 1775307 "CDM Research Exercise 4"    ""           2015-03-06 20:49:00 ""             <NA>       NA                 }
\CommentTok{#> 2 1775308 "Metformin user"             ""           2015-03-06 20:49:00 ""             <NA>       NA                 }
\CommentTok{#> 3 1775309 "Sulfonylurea user"          ""           2015-03-06 20:49:00 ""             <NA>       NA                 }
\CommentTok{#> 4 1775310 "COPY OF: Sulfonylurea user" ""           2015-03-06 20:49:00 ""             <NA>       NA                 }
\CommentTok{#> 5 1775311 "Hypoglycemia "              ""           2015-03-06 20:49:00 ""             <NA>       NA                 }
\CommentTok{#> 6 1775316 "AF (FA_17092020)"           ""           2015-03-06 20:49:00 ""            ""          NA}
\end{Highlighting}
\end{Shaded}

Similar approach may be used for all categories as follows:

\begin{Shaded}
\begin{Highlighting}[]

\NormalTok{ROhdsiWebApi}\OperatorTok{::}\KeywordTok{getDefinitionsMetadata}\NormalTok{(}\DataTypeTok{baseUrl =}\NormalTok{ baseUrl, }
                                     \DataTypeTok{category =} \StringTok{'estimation'}\NormalTok{) }\OperatorTok\StringTok{ }
\StringTok{  }\KeywordTok{arrange}\NormalTok{(.data}\OperatorTok{$}\NormalTok{id) }\OperatorTok\StringTok{ }
\StringTok{  }\KeywordTok{rename_all}\NormalTok{(}\DataTypeTok{.funs =}\NormalTok{ SqlRender}\OperatorTok{::}\NormalTok{camelCaseToTitleCase) }\OperatorTok\StringTok{ }
\StringTok{  }\KeywordTok{tail}\NormalTok{()}
\CommentTok{#> # A tibble: 6 x 6}
\CommentTok{#>      Id Name                  `Created Date`      `Modified Date`     Type        Description                                      }
\CommentTok{#>   <int> <chr>                 <dttm>              <dttm>              <chr>       <chr>                                            }
\CommentTok{#> 1   394 "EXAMPLE : New Popul~ 2020-08-27 09:46:44 2020-08-27 10:36:56 Comparativ~  <NA>                                            }
\CommentTok{#> 2   395 " [NA] 0828 AKI - pr~ 2020-08-27 22:21:09 NA                  Comparativ~  <NA>                                            }
\CommentTok{#> 3   396 "jk_new"              2020-08-28 02:26:52 NA                  Comparativ~ "drug - osteo "                                  }
\CommentTok{#> 4   397 "[1105] DOAC - subgr~ 2020-09-02 12:45:18 2020-09-02 12:45:44 Comparativ~  <NA>                                            }
\CommentTok{#> 5   398 "metformin_combined ~ 2020-09-10 22:23:41 2020-09-10 22:38:04 Comparativ~ "groups with _itt represent intention-to-treat a~}
\CommentTok{#> 6   399 "[PLE Tutorial] Grah~ 2020-09-14 11:49:02 2020-09-15 13:57:36 Comparativ~  <NA>}
\end{Highlighting}
\end{Shaded}

\begin{Shaded}
\begin{Highlighting}[]

\NormalTok{ROhdsiWebApi}\OperatorTok{::}\KeywordTok{getEstimationDefinitionsMetaData}\NormalTok{(}\DataTypeTok{baseUrl =}\NormalTok{ baseUrl) }\OperatorTok\StringTok{ }
\StringTok{  }\KeywordTok{arrange}\NormalTok{(.data}\OperatorTok{$}\NormalTok{id) }\OperatorTok\StringTok{ }
\StringTok{  }\KeywordTok{rename_all}\NormalTok{(}\DataTypeTok{.funs =}\NormalTok{ SqlRender}\OperatorTok{::}\NormalTok{camelCaseToTitleCase) }\OperatorTok\StringTok{ }
\StringTok{  }\KeywordTok{tail}\NormalTok{()}
\CommentTok{#> # A tibble: 6 x 6}
\CommentTok{#>      Id Name                  `Created Date`      `Modified Date`     Type        Description                                      }
\CommentTok{#>   <int> <chr>                 <dttm>              <dttm>              <chr>       <chr>                                            }
\CommentTok{#> 1   394 "EXAMPLE : New Popul~ 2020-08-27 09:46:44 2020-08-27 10:36:56 Comparativ~  <NA>                                            }
\CommentTok{#> 2   395 " [NA] 0828 AKI - pr~ 2020-08-27 22:21:09 NA                  Comparativ~  <NA>                                            }
\CommentTok{#> 3   396 "jk_new"              2020-08-28 02:26:52 NA                  Comparativ~ "drug - osteo "                                  }
\CommentTok{#> 4   397 "[1105] DOAC - subgr~ 2020-09-02 12:45:18 2020-09-02 12:45:44 Comparativ~  <NA>                                            }
\CommentTok{#> 5   398 "metformin_combined ~ 2020-09-10 22:23:41 2020-09-10 22:38:04 Comparativ~ "groups with _itt represent intention-to-treat a~}
\CommentTok{#> 6   399 "[PLE Tutorial] Grah~ 2020-09-14 11:49:02 2020-09-15 13:57:36 Comparativ~  <NA>}
\end{Highlighting}
\end{Shaded}

This is a generic framework that applies to most WebApi categories, and
supports different types of CRUD functionalities like
deleteConceptSetDefinition() vs deleteDefinition(category =
`conceptSet').

\hypertarget{concept-set}{%
\section{Concept Set}\label{concept-set}}

Please review
\href{https://ohdsi.github.io/TheBookOfOhdsi/Cohorts.html\#conceptSets}{`Concept
sets - The Book of OHDSI'}

We commonly post concept set expression into WebApi/Atlas, or try get an
expression from Atlas/WebApi based on a conceptSetDefinitionId.

Example: lets say we have concept set expression as follows, that is
being used for a Rheumatoid Arthritis study.

\begin{Shaded}
\begin{Highlighting}[]

\NormalTok{jsonExpression <-}\StringTok{ '\{}
\StringTok{  "items": [}
\StringTok{    \{}
\StringTok{      "concept": \{}
\StringTok{        "CONCEPT_ID": 81097,}
\StringTok{        "CONCEPT_NAME": "Feltys syndrome",}
\StringTok{        "STANDARD_CONCEPT": "S",}
\StringTok{        "STANDARD_CONCEPT_CAPTION": "Standard",}
\StringTok{        "INVALID_REASON": "V",}
\StringTok{        "INVALID_REASON_CAPTION": "Valid",}
\StringTok{        "CONCEPT_CODE": "57160007",}
\StringTok{        "DOMAIN_ID": "Condition",}
\StringTok{        "VOCABULARY_ID": "SNOMED",}
\StringTok{        "CONCEPT_CLASS_ID": "Clinical Finding"}
\StringTok{      \},}
\StringTok{      "isExcluded": true,}
\StringTok{      "includeDescendants": false,}
\StringTok{      "includeMapped": false}
\StringTok{    \},}
\StringTok{    \{}
\StringTok{      "concept": \{}
\StringTok{        "CONCEPT_ID": 80809,}
\StringTok{        "CONCEPT_NAME": "Rheumatoid arthritis",}
\StringTok{        "STANDARD_CONCEPT": "S",}
\StringTok{        "STANDARD_CONCEPT_CAPTION": "Standard",}
\StringTok{        "INVALID_REASON": "V",}
\StringTok{        "INVALID_REASON_CAPTION": "Valid",}
\StringTok{        "CONCEPT_CODE": "69896004",}
\StringTok{        "DOMAIN_ID": "Condition",}
\StringTok{        "VOCABULARY_ID": "SNOMED",}
\StringTok{        "CONCEPT_CLASS_ID": "Clinical Finding"}
\StringTok{      \},}
\StringTok{      "isExcluded": false,}
\StringTok{      "includeDescendants": true,}
\StringTok{      "includeMapped": false}
\StringTok{    \},}
\StringTok{    \{}
\StringTok{      "concept": \{}
\StringTok{        "CONCEPT_ID": 4035611,}
\StringTok{        "CONCEPT_NAME": "Seropositive rheumatoid arthritis",}
\StringTok{        "STANDARD_CONCEPT": "S",}
\StringTok{        "STANDARD_CONCEPT_CAPTION": "Standard",}
\StringTok{        "INVALID_REASON": "V",}
\StringTok{        "INVALID_REASON_CAPTION": "Valid",}
\StringTok{        "CONCEPT_CODE": "239791005",}
\StringTok{        "DOMAIN_ID": "Condition",}
\StringTok{        "VOCABULARY_ID": "SNOMED",}
\StringTok{        "CONCEPT_CLASS_ID": "Clinical Finding"}
\StringTok{      \},}
\StringTok{      "isExcluded": false,}
\StringTok{      "includeDescendants": true,}
\StringTok{      "includeMapped": false}
\StringTok{    \}}
\StringTok{  ]}
\StringTok{\}'}
\end{Highlighting}
\end{Shaded}

Lets call this concept set expression - `{[}ROhdsiWebApi Vignette{]}
Rheumatoid Arthritis concept set'.

We will need to check if there is a concept set by this name.

\begin{Shaded}
\begin{Highlighting}[]
\CommentTok{# check if there is a concept set by this name, if yes, delete it}
\NormalTok{exists <-}\StringTok{ }\NormalTok{ROhdsiWebApi}\OperatorTok{::}\KeywordTok{existsConceptSetName}\NormalTok{(}\DataTypeTok{conceptSetName =}\NormalTok{ conceptSetName, }\DataTypeTok{baseUrl =}\NormalTok{ baseUrl)}
\NormalTok{exists}
\CommentTok{#> # A tibble: 1 x 6}
\CommentTok{#>   createdBy modifiedBy createdDate         modifiedDate             id name                                                    }
\CommentTok{#>   <chr>     <chr>      <dttm>              <dttm>                <int> <chr>                                                   }
\CommentTok{#> 1 ""        ""         2020-09-17 13:38:25 2020-09-17 13:38:25 1865538 [ROhdsiWebApi Vignette] Rheumatoid Arthritis concept set}
\end{Highlighting}
\end{Shaded}

If there is a concept set with this name, then we could either choose
another name - or delete the old concept set. For this vignette we have
chosen to delete any matching concept set as follows:

\begin{Shaded}
\begin{Highlighting}[]
\ControlFlowTok{if}\NormalTok{ (}\OperatorTok{!}\KeywordTok{isFALSE}\NormalTok{(exists)) \{}
\NormalTok{  ROhdsiWebApi}\OperatorTok{::}\KeywordTok{deleteConceptSetDefinition}\NormalTok{(}\DataTypeTok{conceptSetId =}\NormalTok{ exists}\OperatorTok{$}\NormalTok{id, }\DataTypeTok{baseUrl =}\NormalTok{ baseUrl)}
\NormalTok{\}}
\CommentTok{#> Successfully deleted conceptSet definition id 1865538. Request status code: Success: (204) No Content}
\CommentTok{#> NULL}
\end{Highlighting}
\end{Shaded}

Now we need to ensure the JSON expression above is converted to R-data
object. Note: By design, ROhdsiWebAPi does not accept JSON. It needs to
be converted to R (list) expression

We can now post this R-object into WebApi as follows:

\begin{Shaded}
\begin{Highlighting}[]
\NormalTok{returnFromPostRequest <-}\StringTok{ }\NormalTok{ROhdsiWebApi}\OperatorTok{::}\KeywordTok{postConceptSetDefinition}\NormalTok{(}\DataTypeTok{baseUrl =}\NormalTok{ baseUrl, }
                                                                \DataTypeTok{conceptSetDefinition =}\NormalTok{ rExpression, }
                                                                \DataTypeTok{name =}\NormalTok{ conceptSetName)}
\CommentTok{#> Post ConceptSet definition was successful}
\end{Highlighting}
\end{Shaded}

If successful, we will get a return object as follows into R.

\begin{verbatim}
#> # A tibble: 1 x 6
#>   createdBy modifiedBy createdDate         modifiedDate             id name                                                    
#>   <lgl>     <lgl>      <dttm>              <dttm>                <int> <chr>                                                   
#> 1 NA        NA         2020-09-17 13:40:52 2020-09-17 13:40:52 1865539 [ROhdsiWebApi Vignette] Rheumatoid Arthritis concept set
\end{verbatim}

The id of the newly posted concept-set definition is 1865539. We can now
use this concept-set for many concept set queries eg.,

\hypertarget{if-want-to-print-ready-expression-of-the-concept-set-definition}{%
\subsection{if want to print ready expression of the concept set
definition}\label{if-want-to-print-ready-expression-of-the-concept-set-definition}}

\begin{Shaded}
\begin{Highlighting}[]
\NormalTok{conceptSetDefinition =}\StringTok{ }\KeywordTok{getConceptSetDefinition}\NormalTok{(}\DataTypeTok{conceptSetId =}\NormalTok{ returnFromPostRequest}\OperatorTok{$}\NormalTok{id, }
                                               \DataTypeTok{baseUrl =}\NormalTok{ baseUrl)}
\NormalTok{conceptTbl <-}
\StringTok{  }\KeywordTok{convertConceptSetDefinitionToTable}\NormalTok{(conceptSetDefinition)}
\KeywordTok{names}\NormalTok{(conceptTbl) <-}
\StringTok{  }\NormalTok{SqlRender}\OperatorTok{::}\KeywordTok{camelCaseToTitleCase}\NormalTok{(}\KeywordTok{names}\NormalTok{(conceptTbl))}
\NormalTok{conceptTbl}
\CommentTok{#> # A tibble: 3 x 13}
\CommentTok{#>   `Is Excluded` `Include Descen~ `Include Mapped` `Concept Id` `Concept Name` `Standard Conce~ `Standard Conce~ `Invalid Reason`}
\CommentTok{#>   <lgl>         <lgl>            <lgl>                   <int> <chr>          <chr>            <chr>            <chr>           }
\CommentTok{#> 1 TRUE          FALSE            FALSE                   81097 Felty's syndr~ S                Standard         V               }
\CommentTok{#> 2 FALSE         TRUE             FALSE                   80809 Rheumatoid ar~ S                Standard         V               }
\CommentTok{#> 3 FALSE         TRUE             FALSE                 4035611 Seropositive ~ S                Standard         V               }
\CommentTok{#> # ... with 5 more variables: `Invalid Reason Caption` <chr>, `Concept Code` <chr>, `Domain Id` <chr>, `Vocabulary Id` <chr>,}
\CommentTok{#> #   `Concept Class Id` <chr>}
\end{Highlighting}
\end{Shaded}

\texttt{createConceptSetWorkbook} maybe used to create an Excel workbook
of the concept set.

If we want a list of all conceptId's (including descendants) from the
concept set expression

\begin{Shaded}
\begin{Highlighting}[]
\NormalTok{resolvedConcepts =}\StringTok{ }\KeywordTok{resolveConceptSet}\NormalTok{(}\DataTypeTok{conceptSetDefinition =}\NormalTok{ conceptSetDefinition, }\DataTypeTok{baseUrl =}\NormalTok{ baseUrl)}
\KeywordTok{print}\NormalTok{(}\StringTok{"Note: Showing only the first 10 concept id's"}\NormalTok{)}
\CommentTok{#> [1] "Note: Showing only the first 10 concept id's"}
\NormalTok{resolvedConcepts[}\DecValTok{1}\OperatorTok{:}\DecValTok{10}\NormalTok{]}
\CommentTok{#>  [1]   80809 4035427 4035611 4103516 4114439 4114440 4114441 4114442 4114444 4115050}
\end{Highlighting}
\end{Shaded}

The concept set expression json expression can be recaptured from WebApi
as follows

\begin{Shaded}
\begin{Highlighting}[]
\NormalTok{json <-}
\StringTok{  }\KeywordTok{getConceptSetDefinition}\NormalTok{(}\DataTypeTok{baseUrl =}\NormalTok{ baseUrl, }
                          \DataTypeTok{conceptSetId =}\NormalTok{ returnFromPostRequest}\OperatorTok{$}\NormalTok{id}
\NormalTok{                          )}\OperatorTok{$}\NormalTok{expression }\OperatorTok\StringTok{ }
\StringTok{  }\NormalTok{RJSONIO}\OperatorTok{::}\KeywordTok{toJSON}\NormalTok{(}\DataTypeTok{pretty =} \OtherTok{TRUE}\NormalTok{)}
\end{Highlighting}
\end{Shaded}

Similar framework maybe used with other WebApi categories such as
Cohorts/Characterization/Incidence Rate.

\hypertarget{applications-of-rohdsiwebapi}{%
\section{Applications of
ROhdsiWebAPi}\label{applications-of-rohdsiwebapi}}

A valuable feature of ROhdsiWebApi is that it is able to get full result
set into R, as a data frame object. Results of Cohort, Characterization,
Incidence Rate, Pathway maybe obtained. This data frame may then be
converted to publication ready material by using packages like
\href{https://davidgohel.github.io/officer/}{Officer},
\href{https://davidgohel.github.io/flextable/}{flextable}. The functions
in ROhdsiWebAPi maybe used to create dynamic R-shiny apps that allow
user to interact with WebApi and select cohort definitions, concept sets
for review or modifications. ROhdsiWebApi may be used to build `mini
versions' of Atlas that is project specific - by directly interacting
with WebApi using R.

\hypertarget{cohortscharacterizationincidence-rate}{%
\subsection{Cohorts/Characterization/Incidence
rate}\label{cohortscharacterizationincidence-rate}}

Please review
\href{https://ohdsi.github.io/TheBookOfOhdsi/Cohorts.html\#what-is-a-cohort}{`What
is a cohort - The Book of OHDSI'}.

\begin{quote}
We define a cohort as a set of persons who satisfy one or more inclusion
criteria for a duration of time. The term cohort is often interchanged
with the term phenotype. Cohorts are used throughout OHDSI analytical
tools and network studies as the primary building blocks for executing a
research question.
\end{quote}

\begin{quote}
A cohort is defined as the set of persons satisfying one or more
inclusion criteria for a duration of time. One person may qualify for
one cohort multiple times during non-overlapping time intervals. Cohorts
are constructed in ATLAS by specifying cohort entry criteria and cohort
exit criteria. Cohort entry criteria involve selecting one or more
initial events, which determine the start date for cohort entry, and
optionally specifying additional inclusion criteria which filter to the
qualifying events. Cohort exit criteria are applied to each cohort entry
record to determine the end date when the person's episode no longer
qualifies for the cohort.
\end{quote}

Cohorts/Characterization/Incidence Rate are WebApi categories, where
WebApi manages the execution of generations.

Example: We may want to know if a certain cohort specification has been
generated by checking cohort generation status
\texttt{getCohortGenerationInformation(baseUrl\ =\ baseUrl,\ cohortId=\ 4234)}.
If a cohort is not previously generated, it may be generated using
\texttt{invokeCohortSetGeneration(baseUrl\ =\ baseUrl,\ cohortId\ =\ 4234,\ sourceKey\ =\ \textquotesingle{}HCUP\textquotesingle{})}
. If it is already generated, we can extract its output of cohort
generation using \texttt{getCohortResults(baseUrl,\ cohortId\ =\ 4234)}.

\hypertarget{characterization}{%
\subsection{Characterization}\label{characterization}}

Please review
\href{https://ohdsi.github.io/TheBookOfOhdsi/Characterization.html}{`Characterization
- The Book of OHDSI'}.

\hypertarget{population-level-effect-estimation}{%
\subsection{Population Level Effect
Estimation}\label{population-level-effect-estimation}}

Please review
\href{https://ohdsi.github.io/TheBookOfOhdsi/PopulationLevelEstimation.html}{`Population
Level Effect Estimation - The Book of OHDSI'}.

\hypertarget{patient-level-prediction}{%
\section{Patient Level Prediction}\label{patient-level-prediction}}

Please review
\href{https://ohdsi.github.io/TheBookOfOhdsi/PatientLevelPrediction.html}{`Patient
Level Prediction - The Book of OHDSI'}.

\end{document}
